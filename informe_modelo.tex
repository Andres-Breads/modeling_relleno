\documentclass{article}
\usepackage[utf8]{inputenc}

\title{Complejidad y Optimización \\
Problema del Relleno Sanitario}
\author{Carlos Andrés Aucaruri Correa}
\date{Febrero 16 2022}

\begin{document}

\maketitle

\pagebreak

\tableofcontents

\pagebreak

\section{Modelo}

\subsection{Parámetros}
\begin{itemize}
    \item $m$: representa el número de ciudades
    \item $n$: representa el tamaño de la región cuadrada tal que $n \geq 0$
    \item $Ciudad_ix$: representa la coordenada en $x$ ($0 \leq x \leq n$) de cada ciudad $i$ ($\forall i \in \{1,...,m\}$).
    \item $Ciudad_iy$: representa la coordenada en $y$ ($0 \leq y \leq n$) de cada ciudad $i$ ($\forall i \in \{1,...,m\}$).
\end{itemize}

\subsection{Variables}
\begin{itemize}
    \item $Relleno_x$: representa la coordenada en $x$ del relleno sanitario ($0 \leq x \leq n$).
    \item $Relleno_y$: representa la coordenada en $y$ del relleno sanitario ($0 \leq y \leq n$).
    \item $Distancia$: representa la distancia Manhattan del relleno sanitario a la ciudad más cercana ($0 \leq Distancia \leq n*2$).
    \item $CiudadCercana_x$: representa la coordenada en $x$ de la ciudad más cercana ($0 \leq x \leq n$).
    \item $CiudadCercana_y$: representa la coordenada en $y$ de la ciudad más cercana ($0 \leq y \leq n$).
\end{itemize}

\subsection{Restricciones}
\begin{itemize}
    \item El relleno debe encontrar la distancia más cercana entre todas las ciudades:
    $Distancia = \min( |Relleno_x - Ciudad_ix| + |Relleno_y - Ciudad_iy|, \{ \forall i \in \{1,...,m\} \})$
    \item La ciudad más cercana será la que coincida con la distancia encontrada
\end{itemize}

\subsection{Función Objetivo}
Maximice: $|Relleno_x - CiudadCercana_x| + |Relleno_y - CiudadCercana_y|$

\end{document}