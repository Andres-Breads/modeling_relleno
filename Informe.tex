\documentclass{article}
\usepackage[utf8]{inputenc}
\usepackage{listings}

\title{Complejidad y Optimización \\
Problema del Relleno Sanitario}
\author{Carlos Andrés Aucaruri Correa}
\date{Febrero 16 2022}

\begin{document}

\maketitle

\pagebreak

\tableofcontents

\pagebreak

\section{Modelo}
La región EcoReg tiene un problema serio de depósito de basuras, y ha decidido construir un nuevo relleno dentro de sus fronteras. Como es natural,
cada ciudad dentro de la región está en alerta y presionando para que el sitio no quede cerca de su ciudad. Por tal razón, los administradores de la región
quieren encontrar un sitio que quede lo más lejos posible de la ciudad más cercana. Los administradores
han decidido medir la distancia entre dos ciudades con la métrica Manhattan la cual define la distancia
entre dos puntos como la distancia en el eje X más la distancia en el eje Y.

La región se representa como un cuadrado perfecto de N km por N km. Identificamos la esquina al suroccidente de la región con la posición (0,0).
En este sistema, las ciudades están situadas sobre las intersecciones.

\subsection{Parámetros}
\begin{itemize}
    \item $m$: representa el número de ciudades
    \item $n$: representa el tamaño de la región cuadrada tal que $n \geq 0$
    \item $Ciudad_ix$: representa la coordenada en $x$ ($0 \leq x \leq n$) de cada ciudad $i$ ($\forall i \in \{1,...,m\}$).
    \item $Ciudad_iy$: representa la coordenada en $y$ ($0 \leq y \leq n$) de cada ciudad $i$ ($\forall i \in \{1,...,m\}$).
\end{itemize}

\subsection{Variables}
\begin{itemize}
    \item $Relleno_x$: representa la coordenada en $x$ del relleno sanitario ($0 \leq x \leq n$).
    \item $Relleno_y$: representa la coordenada en $y$ del relleno sanitario ($0 \leq y \leq n$).
    \item $Distancia$: representa la distancia Manhattan del relleno sanitario a la ciudad más cercana ($0 \leq Distancia \leq n*2$).
    \item $CiudadCercana_x$: representa la coordenada en $x$ de la ciudad más cercana ($0 \leq x \leq n$).
    \item $CiudadCercana_y$: representa la coordenada en $y$ de la ciudad más cercana ($0 \leq y \leq n$).
\end{itemize}

\subsection{Restricciones}
\begin{itemize}
    \item El relleno debe encontrar la distancia más cercana entre todas las ciudades:
    $Distancia = \min( |Ciudad_ix - Relleno_x| + |Ciudad_iy - Relleno_y|, \{ \forall i \in \{1,...,m\} \})$
    \item La ciudad más cercana será la que coincida con la distancia encontrada
    Para minizinc tendremos las siguientes Restricciones usando la restriccion global "table.mzn":
    \begin{lstlisting}
    constraint table(ciudad_cercana, ciudades);
    constraint abs(ciudad_cercana[1] - relleno_x) +
        abs(ciudad_cercana[2] - relleno_y) == distancia;
    \end{lstlisting}
\end{itemize}

\subsection{Función Objetivo}
Maximice: $Distancia$

\section{Detalles importantes de implementación:}

\end{document}